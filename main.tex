\documentclass{article}

\usepackage[margin=1.5cm, includefoot, footskip=30pt]{geometry}

\usepackage[T1]{fontenc}
\usepackage{amsmath}
\usepackage{booktabs}
\usepackage{graphicx}
\usepackage{hyperref}
\usepackage{listings}
\usepackage{lmodern}
\usepackage{multicol}
\usepackage{standalone}
\usepackage{xcolor}
\usepackage{subcaption}

\usepackage{tikz}
\usetikzlibrary{calc, shapes, patterns}

\title{Reviving, reproducing and revisiting Axelrod's second tournamentt}
\author{Vincent Knight \and Owen Campbell \and Marc Harper \and T. J. Gaffney \and Nikoleta Glynatsi}

\begin{document}

\maketitle

\section{Introduction}\label{sec:introduction}

The Prisoner's Dilemma has been the centre of the study of cooperative behaviour
since Robert Axelrod's seminal work in the 1980s~\cite{Axelrod1980a,
Axelrod1980b, Axelrodbook}. In this work, Robert Axelrod invited fellow
academics and the wider public, through computer hobbyist magazines (one
contribution was from an eleven year old), to write computer code to play the
Iterated Prisoner's Dilemma.

This initial research has subsequently led to a huge area
of research aiming to understand the emergence of cooperative behaviour. A good
overview of the variety of research areas is given in~\cite{Nowak2006}.

In the Prisoner's Dilemma, each player chooses
simultaneously and independently
between cooperation (C) or defection (D). The payoffs of
the game are defined by the matrix
\(
\begin{pmatrix}
    R & S \\ 
    T & P
\end{pmatrix}
\),
where $T > R > P > S$ and $2R > T + S$. The PD is a one
round game, but is commonly studied in a manner where the prior outcomes
matter. This repeated form is called the Iterated Prisoner's
Dilemma (IPD).


In Robert Axelrod's tournaments the particular values \(R=3, P=1, S=0, T=5\)
were used.  Strategies were submitted written in either
Fortran~\cite{smith1994programming} or Basic~\cite{Bishop2004}
languages (in which case they were translated to Fortran). The very first
tournament~\cite{Axelrod1980a} involved 14 submissions  (complemented by a
15th random strategy). Famously, the winner of this tournament was Tit For
Tat: a strategy that mimics the opponent's previous action. The only sources
for this work are the paper itself, all of the source code is apparently lost.
In the second tournament~\cite{Axelrod1980b} 64 strategies were submitted and
again: Tit For Tat was declared the winner. From a computation archaeological
point of view this tournament was far superior as the source code for all the
strategies was kept and made available for use. It can be found at
\url{http://www-personal.umich.edu/~axe/research/Software/CC/CC2/TourExec1.1.f.html}.

This manuscript describes the process of reviving and using these strategies in
a modern software framework (\cite{AxelrodProject}): a Python library with over
200 strategies and a large number of analytical procedures which has been used
in ongoing research~\cite{Harper2017, Knight2017, glynatsi2024properties,
bucciarelli2024studying, macmillan2024game}.

Importantly, reviving the strategies is not done
through a manual exercise of reverse engineering the Fortran code which would be
prone to mistakes. This is done by calling the original functions ensuring the
\textbf{best possible reproduction and analysis of Robert Axelrod's original
work}. This will be described in more detail in Section~\ref{sec:reviving}.

Results and further analyses pertaining to reproducing the tournament will be
given in Section~\ref{sec:reproducing}. Finally, Section~\ref{sec:revisiting}
will extend the analysis to include contemporary research:

\begin{itemize}
    \item Experimenting with adding one, two or three of over 200 other strategies
        from~\cite{AxelrodProject} to see how the results change.
    \item Running a tournament with all considered strategies (more than 250).
\end{itemize}

This work contributes to the game theoretic literature by providing the first
exercise in reproducing reported results that have been at the core of the study
of cooperation. Furthermore it also provides a contemporary lens which, amongst
other things concludes with one of the largest Iterated Prisoner's Dilemma
tournaments.
Finally, by reviving the original code and making it available to use
in~\cite{AxelrodProject}~it is now possible to use the original strategies of
Robert Axelrod's second tournament in a modern framework which for example
allows for the study of topological and evolutionary variants.

\section{Reviving the tournament}\label{sec:reviving}

As described in Section~\ref{sec:introduction}, the original source code for
Axelrod's second tournament was written in Fortan (some contributors submitted
code in Basic), this was subsequently published at~\cite{Axelrod1980bCode}. This
website maintained by the University of Michigan Center for the Study of Complex
Systems was last updated (at the time of writing) in 1996. The source code was
originally a single file (TourExec1.1.f) and is published on the site in HTML format.

For the purposes of this work, the html formatting was removed to produce the original
fortran file which was then minimally modified so that it would compile on a modern
compiler.

Furthermore, each strategy was extracted in to a single modular file which
follows modern best practice and makes analysis more readable.

Finally, a Makefile was created to control the compilation of the fortran
strategy files into a single binary shared object file (named
\texttt{libstrategies.so}) which is then installed into a standard location on
a Posix compliant operating system.

This work can be found
at \url{https://github.com/Axelrod-Python/TourExec} and has been archived
at~\cite{TourExec}. This archival is itself a major contribution of this paper
as it ensures permanency of the original source code.

A Python library has been written that enables an interface to
the Axelrod library described in the previous section. This library is referred
to as the \texttt{Axelrod\_fortran} library and is available
at \url{https://github.com/Axelrod-Python/axelrod-fortran} and the specific
version used for this work is~\cite{Axelrod_fortran}.

This library has the binary file \texttt{libstrategies.so} described above as
a dependency but otherwise offers a straightforward to install and use option
for the study of the strategies of~\cite{Axelrod1980b} (using standard
scientific Python packages).

For this to work there are a variety of minor translations that need to take
place. As documented at
\url{http://www-personal.umich.edu/~axe/research/Software/CC/CC2/TourExec1.1.f.html}
the Fortran code implies that each strategy is a Fortran function which takes
the following inputs.

\begin{itemize}
    \item  \texttt{J}: The opponent's previous move: 0 corresponds to a
        defection and 1 to a cooperation.
    \item \texttt{M}: The current turn number (starting at 1).
    \item \texttt{K}: The player's current cumulative score.
    \item \texttt{L}: The opponent's current cumulative score.
    \item \texttt{R}: A random number between 0 and 1: used by stochastic
        strategies.
    \item \texttt{JA}: The player's previous move.
\end{itemize}

For example, Figure~\ref{fig:tft_fortran} shows the source code for
\texttt{k92r.f} also known as Tit For Tat.

\begin{figure}[!hbtp]
    \begin{center}
        \begin{lstlisting}[language=Fortran,
                           basicstyle=\ttfamily,
                           frame=single,
                           keywordstyle=\color{red},
                           commentstyle=\color{green}]
          FUNCTION K92R(J,M,K,L,R, JA)
    C BY ANATOL RAPOPORT
    C TYPED BY AX 3/27/79 (SAME AS ROUND ONE TIT FOR TAT)
    c replaced by actual code, Ax 7/27/93
    c  T=0
    c   K92R=ITFTR(J,M,K,L,T,R)
          k92r=0
          k92r = j
    c test 7/30
    c   write(6,77) j, k92r
    c77   format(' test k92r. j,k92r: ', 2i3)
          RETURN
          END
        \end{lstlisting}
        \caption{Fortran Source code for original Tit For Tat strategy submitted
        to Axelrod's second tournament.}
        \label{fig:tft_fortran}
    \end{center}
\end{figure}


The \texttt{Axelrod} library takes advantage of the modern Object Oriented
framework in Python. Each strategy is a class with agent based behaviour. The
input to each player is simply the opponent with both holding their respective
history of plays. In the newly written \texttt{Axelrod\_fortran} library a
class inherited from the base \texttt{Axelrod} class is written that interfaces
with the Fortran strategies and the Python requirements. This includes, for
example passing an initial move required by the Fortran player: using the
original Fortran code as reference (see Figure~\ref{fig:tournament_code}) it
is assumed that the assumed prior move is
a cooperation (which is in line with Figure~\ref{fig:tft_fortran}).

\begin{figure}[!hbtp]
    \begin{center}
        \begin{lstlisting}[language=Fortran,
                           basicstyle=\ttfamily,
                           frame=single,
                           keywordstyle=\color{red},
                           numbers=left,
                           firstnumber=67,
                           commentstyle=\color{green}]
      Do 20 Game = 1,5
            RowGameSc = 0
            ColGameSc = 0
            JA = 0          ! Row's previous move, reported to column
            JB = 0          ! Col's previous move, reported to row
        \end{lstlisting}
        \caption{A portion of the code
            \url{https://github.com/Axelrod-Python/TourExec/blob/master/src/tournament/AxTest.f}
            setting the default
        previous move to a cooperation and score to 0.}
        \label{fig:tournament_code}
    \end{center}
\end{figure}

Figure~\ref{fig:strategy_diagram} shows a diagrammatic representation of the
interface between the Python strategy and the Fortran function.

\begin{figure}[!hbtp]
    \begin{center}
        \scalebox{.8}{
            \includestandalone{assets/strategy_diagram}
                     }
        \caption{The interface between the Python axelrod library and the
				 Fortran code}
        \label{fig:strategy_diagram}
    \end{center}
\end{figure}

The major advantage of this approach is that at no point has any subjectivity
been added to the process of replicating Axelrod's second tournament. Indeed for
some strategies the only description available is the Fortran code itself, thus
they are being run and used as available. To the authors' knowledge this is the
best possible way to replicate Axelrod's work which is the subject of the next
section.

\section{Reproducing the tournament}\label{sec:reproducing}

From~\cite{Axelrod1980b}, the following characteristics of the original
tournament have been identified:

\begin{itemize}
    \item Matches have length from \(\{63, 77, 151, 308, 156\}\);
    \item Players do not know the number of rounds in a given match;
    \item A total of \input{assets/number_of_original_repetitions.tex}repetitions
        across the various match lengths have been carried out.
\end{itemize}

Note that there is some lack of clarity in~\cite{Axelrod1980b} as to the length
of the matches:

\begin{quote}
    \textit{``As announced in the rules, the length of the games was determined
        probabilistically with a .00346 chance of ending with each given move.
        This parameter was chosen so that the expected median length of a game
        would be 200 moves. In practice, each pair of players was matched five
        times, and the lengths of these five games were determined once and for
        all by drawing a random sample. The resulting random sample from the
        implied distribution specified that the five games for each pair of
        players would be of lengths 63, 77, 151, and 308 moves. Thus the average
    length of a game turned out to be somewhat shorter than expected at 151
moves.''}
\end{quote}

As only four match length samples were specified but the average was given, a
fifth match length of 156 is assumed (giving the correct average of 151).  It
seems that the only stochastic smoothing used was these five repetitions,
without a known seed it is not possible to replicate, thus the original
tournament is repeated a total of
\input{assets/number_of_original_repetitions.tex}for each match length.

The replicated scores and corresponding rankings of each strategy across all
repetitions are shown in
Figure~\ref{fig:replicated_tournament}.

\begin{figure}[!hbtp]
    \centering
    \begin{subfigure}[t]{.5\textwidth}
        \includegraphics[width=.95\textwidth]{assets/original_tournament_scores.pdf}
        \caption{Average score per turn of each strategy.}
        \label{fig:original_tournament_scores}
    \end{subfigure}%
    ~
    \begin{subfigure}[t]{0.5\textwidth}
        \includegraphics[width=.95\textwidth]{assets/original_tournament_rankings.pdf}
        \caption{Ranking of each strategy.}
        \label{fig:original_tournament_ranks}
    \end{subfigure}
    \caption{Replicated tournament with strategies ordered by original rank}
    \label{fig:replicated_tournament}
\end{figure}

The top 15 strategies in the reproduced tournament are shown in
Table~\ref{tbl:original_tournament_rankings}.

\begin{table}[!hbtp]
        \centering
        \begin{tabular}{llrrr}
\toprule
{} &                                 Author &  Scores &  Rank &  Original Rank \\
\midrule
k92r &                        Anatol Rapoport &  2.8785 &     1 &              1 \\
k61r &                       Danny C Champion &  2.7189 &    17 &              2 \\
k42r &                          Otto Borufsen &  2.8342 &     2 &              3 \\
k49r &                               Rob Cave &  2.8261 &     4 &              4 \\
k44r &                          William Adams &  2.8262 &     3 &              5 \\
k60r &           Jim Graaskamp and Ken Katzen &  2.8102 &     9 &              6 \\
k41r &                            Herb Weiner &  2.8106 &     8 &              7 \\
k75r &                      Paul D Harrington &  2.8241 &     5 &              8 \\
k84r &  T Nicolaus Tideman and Paula Chieruzz &  2.8125 &     7 &              9 \\
k32r &                       Charles Kluepfel &  2.8166 &     6 &             10 \\
k35r &                        Abraham Getzler &  2.8018 &    11 &             11 \\
k68r &                       Fransois Leyvraz &  2.8090 &    10 &             12 \\
k72r &                      Edward C White Jr &  2.7763 &    12 &             13 \\
k46r &                     Graham J Eatherley &  2.7721 &    13 &             14 \\
k83r &                           Paul E Black &  2.7427 &    14 &             15 \\
\bottomrule
\end{tabular}

        \caption{Top 15 strategies in the reproduced tournament}
        \label{tbl:original_tournament_rankings}
\end{table}

Whilst the results show an overall agreement with the original reported results,
a distinct outlier is \texttt{k61r}.
\texttt{k61r}, referred to as Champion: cooperates for the first 10
moves, plays tit for tat for the next fifteen and then will cooperate
unless: the other player defected on the previous move, the
other player cooperated less than 60\% and a random number between
0 and 1 is greater that the other player's cooperation rate.
Upon closer investigation a bug was noted in the original code for
\texttt{k61r}. One initial value was not being initialised, the modified version
of this strategy is shown in Figure~\ref{fig:k61r}

\begin{figure}[!hbtp]
    \begin{center}
        \begin{lstlisting}[language=Fortran,
                           basicstyle=\ttfamily,
                           frame=single,
                           keywordstyle=\color{red},
                           numbers=left,
                           commentstyle=\color{green}]
      FUNCTION K61R(ISPICK,ITURN,K,L,R, JA)
C BY DANNY C. CHAMPION
C TYPED BY JM 3/27/79
      k61r=ja    ! Added 7/27/93 to report own old value
      IF (ITURN .EQ. 1) ICOOP = 0  ! Added 10/8/2017 to fix bug for multiple runs
      IF (ITURN .EQ. 1) K61R = 0
      IF (ISPICK .EQ. 0) ICOOP = ICOOP + 1
      IF (ITURN .LE. 10) RETURN
      K61R = ISPICK
      IF (ITURN .LE. 25) RETURN
      K61R = 0
      COPRAT = FLOAT(ICOOP) / FLOAT(ITURN)
      IF (ISPICK .EQ. 1 .AND. COPRAT .LT. .6 .AND. R .GT. COPRAT)
     +K61R = 1
      RETURN
      END
        \end{lstlisting}
        \caption{Original code for \texttt{k61r} with fixed bug on line 5.}
        \label{fig:k61r}
    \end{center}
\end{figure}

It is not clear if this bug affected the tournament results reported in 1980.
There is no source of the specific code used to run the tournaments and
the bug only effects \texttt{k61r} on a second use of the function (the very
first time: \texttt{ICOOP} is assumed to be 0). Thus, it is possible that every
single match of the tournament was run in isolation.

Despite fixing this bug and verifying all other strategies for potentially
similar bugs there is still a discrepancy in the results. 
Figure~\ref{fig:original_tournament_rankings_all_approaches} shows
the results of the tournament computed using a number of different approaches.

\begin{figure}[!hbtp]
    \begin{center}
        \includegraphics[width=.95\textwidth]{assets/original_tournament_rankings_all_approaches.pdf}
        \caption{Using a variety of approaches to reproduce the original
        tournament. This includes using a Python implementation of Champion's
        strategy as well as using as many Python implementation of the Fortran
        strategies as possible.}
        \label{fig:original_tournament_rankings_all_approaches}
    \end{center}
\end{figure}

There is no immediate explanation for the remaining discrepancy with the results
reported 
in~\cite{Axelrod1980b}. Potential explanations include:

\begin{itemize}
    \item Stochastic variation not being sufficiently taken in to account
        in~\cite{Axelrod1980b}.
    \item A difference with how an older Fortran compiler would interpret the
        commands: this is not obvious though, the implemented version seems to
        interact as expected.
    \item An error in the reporting of~\cite{Axelrod1980b} which could include
        a modification of the source code.
\end{itemize}

Apart from the strategy by Champion, the agreement between the original and the
reproduced tournament is strong. The main conclusions included for example that
Tit For Tat (\texttt{k92r}) once again wins the tournament. Furthermore, the
fact that high performing strategies are ``nice'' is also evident although
perhaps interestingly \texttt{k42r} which takes the second rank of \texttt{61r}
is a strategy that cooperates with most strategies apart from itself.  The
overall cooperation rate of the tournament is
\input{assets/original_tournament_overall_cooperation_rate.tex}.
Figure~\ref{fig:replicated_cooperation_rates} shows the cooperation rates of
the tournament. It is clear that the high performing strategies cooperate
overall more often. Looking at the pairwise cooperation rates in
Figure~\ref{fig:original_tournament_pairwise_cooperation_rates} shows that the
high performing strategies generally seems to cooperate with high performing
strategies. This underpins one of the main conclusions of~\cite{Axelrod1980b}
explaining the emergence of cooperation in competitive environments.

\begin{figure}[!hbtp]
    \begin{subfigure}{.6\textwidth}
        \centering
        \includegraphics[width=.9\textwidth]{assets/original_tournament_cooperation_rate_versus_rank.pdf}
        \caption{Mean cooperation rates}
        \label{fig:original_tournament_cooperation_rate_versus_rank}
    \end{subfigure}%
    ~
    \begin{subfigure}{.4\textwidth}
        \centering
        \includegraphics[width=.9\textwidth]{assets/original_tournament_pairwise_cooperation_rates.pdf}
        \caption{Cooperation rates between each pair of players}
        \label{fig:original_tournament_pairwise_cooperation_rates}
    \end{subfigure}
    \caption{Replicated tournament cooperation rates with strategies
             ordered by original rank}
    \label{fig:replicated_cooperation_rates}
\end{figure}

In~\cite{Axelrod1980b}, a linear regression model is used to identify 5
strategies, the scores against which are good predictors of the overall
performance. The reported \(R^2\) value is \(0.979\) (indicating 97\% of
variance accounted for by the model). For completeness, the coefficients of
this model (reported in~\cite{Axelrod1980b}) are shown in
Table~\ref{tbl:original_tournament_representative_model}.

\begin{table}[!hbtp]
        \centering
        \begin{tabular}{lr}
\toprule
Strategies & Coefficients \\
\midrule
k69r & 0.202000 \\
k91r & 0.198000 \\
k40r & 0.110000 \\
k76r & 0.072000 \\
k67r & 0.086000 \\
Intercept & 0.795000 \\
\bottomrule
\end{tabular}

        \caption{Linear model described in~\cite{Axelrod1980b} with
             \(R^2=\protect\input{assets/original_tournament_representative_r_squared.tex}\)}
        \label{tbl:original_tournament_representative_model}
\end{table}

Given the discrepancy in results shown in
Figure~\ref{fig:replicated_tournament} and
Table~\ref{tbl:original_tournament_rankings} it is not surprising to see that
this model no longer performs as well with
\(R^2=\input{assets/original_tournament_representative_r_squared.tex}\).

Fitting a new model to the same 5 strategies gives the coefficients shown in
Table~\ref{tbl:original_tournament_predictive_with_axelrod_5_model} with
\(R^2=\input{assets/original_tournament_predictive_5_r_squared.tex}\)

\begin{table}[!hbtp]
        \centering
        \begin{tabular}{lrll}
\toprule
Strategies &  Coefficients &    $p$-value & $F$-value \\
\midrule
      k69r &         0.099 &   9.5178e-09 &   44.1166 \\
      k91r &         0.206 &  3.07113e-10 &   56.3992 \\
      k40r &         0.206 &  1.55822e-12 &   78.2667 \\
      k76r &         0.060 &    0.0257417 &   5.22576 \\
      k67r &         0.124 &   1.7738e-06 &   27.9476 \\
 Intercept &         0.770 &           NA &        NA \\
\bottomrule
\end{tabular}

        \caption{Linear model fitted to the same 5 strategies described
                 in~\cite{Axelrod1980b} with
             \(R^2=\protect\input{assets/original_tournament_predictive_with_axelrod_5_r_squared.tex}\)}
        \label{tbl:original_tournament_predictive_with_axelrod_5_model}
\end{table}

Using recursive feature elimination~\cite{guyon2002gene} it is
possible to select the features (strategies) that give the best prediction for
a given number of features. This \textit{best} \(R^2\) versus the number of
features is shown in
Figure~\ref{fig:original_tournament_r_squared_versus_number_of_features}.

\begin{figure}[!hbtp]
    \centering
    \includegraphics[width=.8\textwidth]{assets/original_tournament_r_squared_versus_number_of_features.pdf}
    \caption{\(R^2\) for models obtained using recursive feature elimination.}
    \label{fig:original_tournament_r_squared_versus_number_of_features}
\end{figure}

Tables~\ref{tbl:original_tournament_predictive_5_model},~\ref{tbl:original_tournament_predictive_8_model}
and~\ref{tbl:original_tournament_predictive_25_model} show the coefficients for
linear models fitted to 5, 8 and 25 strategies with
\(R^2=\input{assets/original_tournament_predictive_5_r_squared.tex}\),
\(R^2=\input{assets/original_tournament_predictive_8_r_squared.tex}\) and
\(R^2=\input{assets/original_tournament_predictive_25_r_squared.tex}\)
respectively (8 strategies is the smallest number of strategies for which \(R^2
> 96\) and 25 strategies is the smallest number of strategies for which
\(R^2>99\)).


\begin{table}[!hbtp]
        \centering
        \begin{tabular}{lrll}
\toprule
Strategies &  Coefficients &    $p$-value & $F$-value \\
\midrule
      k46r &         0.252 &   0.00574885 &   8.19518 \\
      k53r &         0.130 &  2.40311e-08 &   41.0388 \\
      k56r &         0.154 &  1.40719e-09 &     50.78 \\
      k82r &         0.189 &  1.92039e-06 &   27.7236 \\
      k85r &         0.109 &  1.81982e-12 &   77.5683 \\
 Intercept &         0.398 &           NA &        NA \\
\bottomrule
\end{tabular}

        \caption{Linear model best fitted to 5 strategies in the reproduced tournament
                 with
             \(R^2=\protect\input{assets/original_tournament_predictive_5_r_squared.tex}\)}
        \label{tbl:original_tournament_predictive_5_model}
\end{table}

\begin{table}[!hbtp]
        \centering
        \begin{tabular}{lrll}
\toprule
Strategies & Coefficients & $p$-value & $F$-value \\
\midrule
k36r & 0.133320 & 0.189994 & 1.756582 \\
k51r & 0.187270 & 0.002438 & 10.001032 \\
k56r & -11.100380 & 0.000000 & 48.729856 \\
k59r & 11.231820 & 0.000000 & 48.744926 \\
k62r & 0.125620 & 0.703116 & 0.146621 \\
k66r & 0.151600 & 0.000000 & 135.614172 \\
k70r & 0.221780 & 0.000000 & 79.384616 \\
k89r & 0.172540 & 0.015430 & 6.210811 \\
Intercept & -0.208980 & NA & NA \\
\bottomrule
\end{tabular}

        \caption{Linear model best fitted to 8 strategies in the reproduced tournament
                 with
             \(R^2=\protect\input{assets/original_tournament_predictive_8_r_squared.tex}\)}
        \label{tbl:original_tournament_predictive_8_model}
\end{table}

\begin{table}[!hbtp]
        \centering
        \begin{tabular}{lrll}
\toprule
Strategies & Coefficients & $p$-value & $F$-value \\
\midrule
k32r & -0.060330 & 0.000000 & 186.247539 \\
k34r & 0.036030 & 0.000000 & 42.866846 \\
k36r & 0.044880 & 0.189994 & 1.756582 \\
k41r & 0.073800 & 0.000000 & 177.937911 \\
k42r & 0.044080 & 0.000000 & 79.351204 \\
k43r & 0.014170 & 0.000000 & 97.073236 \\
k46r & 0.044870 & 0.004805 & 8.566798 \\
k47r & 0.028380 & 0.000000 & 43.355694 \\
k48r & 0.029490 & 0.007426 & 7.670719 \\
k51r & 0.074000 & 0.002438 & 10.001032 \\
k56r & -3.499880 & 0.000000 & 48.729856 \\
k59r & 3.551580 & 0.000000 & 48.744926 \\
k60r & 0.042590 & 0.000000 & 298.163936 \\
k62r & 0.042710 & 0.703116 & 0.146621 \\
k66r & 0.086580 & 0.000000 & 135.614172 \\
k70r & 0.067690 & 0.000000 & 79.384616 \\
k73r & 0.075500 & 0.000000 & 67.006588 \\
k74r & 0.023020 & 0.244116 & 1.383261 \\
k75r & 0.024290 & 0.169382 & 1.933926 \\
k76r & 0.031590 & 0.025992 & 5.207451 \\
k81r & 0.000130 & 0.000000 & 84.798727 \\
k82r & 0.032720 & 0.000001 & 29.385307 \\
k89r & 0.022020 & 0.015430 & 6.210811 \\
k90r & 0.041170 & 0.960298 & 0.002498 \\
k91r & 0.071910 & 0.000000 & 56.667196 \\
Intercept & 0.157940 & NA & NA \\
\bottomrule
\end{tabular}

        \caption{Linear model best fitted to 25 strategies in the reproduced tournament
                 with
             \(R^2=\protect\input{assets/original_tournament_predictive_25_r_squared.tex}\)}
        \label{tbl:original_tournament_predictive_25_model}
\end{table}

The predictions of these models are shown in
Figure~\ref{fig:original_tournament_predictive_score_models}.

\begin{figure}[!hbtp]
    \centering
    \includegraphics[width=.9\textwidth]{assets/original_tournament_predictive_score_models.pdf}
    \caption{Predicting the performance of strategies using the 4 models
             discussed}
    \label{fig:original_tournament_predictive_score_models}
\end{figure}

It is clear that the effectiveness of the predictive models with 5 strategies is
low for the cluster of highly performing strategies (with a score greater than
2.5). To be able to obtain a good model even for high performing strategies 8
seem to provide a good predictive model.

This is the first known re run of the work of~\cite{Axelrod1980b} (which lead to
\cite{Axelrodbook} which has over 33000 citations). Despite some misalignment
with the results the overall principles are well aligned: Tit For Tat wins,
offering scope for the effectiveness of cooperation.~\cite{Axelrod1980b} used
these results to give guidelines for effective behaviour:

\begin{enumerate}
    \item Do not be envious by striving for a payoff larger than the opponent's payoff.
    \item Be ``nice'' by not being the first to defect.
    \item Reciprocate both cooperation and defection; Be provocable to retaliation and forgiveness.
    \item Do not be too clever by scheming to exploit the opponent.
\end{enumerate}

Whilst these certainly apply to Axelrod's particular tournament, in the next
section we revisit the tournament and aim to see
how correct the Game Theoretic community was to take this guidelines as a
generalization.

\section{Revisiting the tournament}\label{sec:revisiting}

In this section, the tournament of~\cite{Axelrod1980b} will be expanded.
Indeed, a large amount of research has gone on since Axelrod's original work
which include for example training of strategies using reinforcement
learning~\cite{Harper2017}
but also the discovery of Zero Determinant strategies~\cite{Press2012}. This
section aims to measure how well these strategies would have faired and
\textbf{if} any of the original insights and conclusions would differ.

\subsection{Running a large tournament}\label{sec:run_with_everyone}

The Axelrod Python library has access to more than 200 strategies and regularly
runs the largest iterated prisoners Dilemma tournament ever run (everytime a new
strategy is contributed to the project the results are updated).
We here expand this tournament further by including all of the original fortran
strategies (omitting duplication). This gives a tournament with
\input{assets/number_of_strategies_in_full_tournament.tex} strategies. As
for~\cite{TourExec} the archival of all of the interactions and results from
this large tournament are another contribution of this paper.

Table~\ref{tbl:full_tournament_rankings}
shows the rankings of the top 20 strategies when including all strategies over
\input{assets/number_of_full_repetitions.tex}repetitions.
The cooperation rate for the tournament that pits all the library strategies
against each other (without the Fortran ones) is
\input{assets/library_tournament_cooperation_rate.tex}\unskip.

\begin{table}[!hbtp]
        \centering
        \scriptsize
        \begin{tabular}{llrrrrl}
\toprule
{} &                     Original Author &  Scores &  Rank &  Library Rank &  Original Rank & Reproduced Rank \\
\midrule
EvolvedLookerUp2\_2\_2       &                                  NA &  2.8397 &     1 &             1 &             NA &              NA \\
Evolved HMM 5              &                                  NA &  2.8278 &     2 &             2 &             NA &              NA \\
Evolved FSM 16 Noise 05    &                                  NA &  2.8117 &     3 &             5 &             NA &              NA \\
Evolved FSM 16             &                                  NA &  2.8073 &     4 &             4 &             NA &              NA \\
PSO Gambler 2\_2\_2          &                                  NA &  2.8057 &     5 &             3 &             NA &              NA \\
Evolved ANN                &                                  NA &  2.7951 &     6 &             7 &             NA &              NA \\
Evolved ANN 5              &                                  NA &  2.7939 &     7 &             6 &             NA &              NA \\
Omega TFT                  &                                  NA &  2.7911 &     8 &            16 &             NA &              NA \\
PSO Gambler Mem1           &                                  NA &  2.7877 &     9 &             9 &             NA &              NA \\
Evolved FSM 4              &                                  NA &  2.7841 &    10 &            10 &             NA &              NA \\
PSO Gambler 1\_1\_1          &                                  NA &  2.7833 &    11 &             8 &             NA &              NA \\
PSO Gambler 2\_2\_2 Noise 05 &                                  NA &  2.7760 &    12 &            19 &             NA &              NA \\
Gradual                    &                                  NA &  2.7695 &    13 &            20 &             NA &              NA \\
Evolved ANN 5 Noise 05     &                                  NA &  2.7669 &    14 &            11 &             NA &              NA \\
DBS                        &                                  NA &  2.7659 &    15 &            12 &             NA &              NA \\
Winner12                   &                                  NA &  2.7628 &    16 &            13 &             NA &              NA \\
k85r                       &  Robert B Falk and James M Langsted &  2.7471 &    17 &            NA &             33 &              36 \\
Spiteful Tit For Tat       &                                  NA &  2.7375 &    18 &            29 &             NA &              NA \\
k80r                       &       Robyn M Dawes and Mark Batell &  2.7370 &    19 &            NA &             36 &              34 \\
k87r                       &                     E E H Schurmann &  2.7315 &    20 &            NA &             44 &              38 \\
\bottomrule
\end{tabular}

        \caption{Top 25 strategies in the tournament when using all available
        strategies}
        \label{tbl:full_tournament_rankings}
\end{table}

The overall cooperation rate of this tournament is
\input{assets/full_tournament_overall_cooperation_rate.tex}and the various
cooperation rates are shown in
Figure~\ref{fig:full_tournament_cooperation_rate_versus_rank} shows the
cooperation rates of each strategy (ordered by rank).

\begin{figure}[!hbtp]
    \centering
    \includegraphics[width=.8\textwidth]{assets/full_tournament_cooperation_rate_versus_rank.pdf}
    \caption{Cooperation rate versus rank for tournament with all available
    strategies}
    \label{fig:full_tournament_cooperation_rate_versus_rank}
\end{figure}

\begin{figure}[!hbtp]
    \centering
    \includegraphics[width=.8\textwidth]{assets/full_tournament_cooperation_rates.pdf}
    \caption{Distribution of cooperation rates for the full tournament.}
    \label{fig:full_tournament_cooperation_rates}
\end{figure}

Figure~\ref{fig:relative_cooperation_rates} shows the change of behaviour
between tournaments. It is clear that the original fortran strategies cooperate
less than they did in the original tournament. Furthermore, looking at Table~\ref{tbl:full_tournament_rankings}
the strategies that rank highly are sophisticated strategies that have been
trained using reinforcement learning. The ``Evolved'' strategies have been
trained using evolutionary algorithms and the ``PSO'' strategies have been
trained using Particle Swarm Algorithms, these are described
in~\cite{Harper2017}. Notably these high performing strategies do not
necessarily follow the Axelrod's guidelines. In fact they are more in line with
the guidelines identified in~\cite{glynatsi2024properties} in which not just one
tournament was used but thousands to identify the following properties:

\begin{enumerate}
    \item Be a little bit envious
    \item Be ``nice'' in non-noisy environments or when game lengths are longer
    \item Reciprocate both cooperation and defection appropriately; Be provocable in tournaments with short matches, and generous in tournaments with noise
    \item It is ok to be clever
    \item Adapt to the environment; Adjust to the mean population cooperation
\end{enumerate}

\begin{figure}[!hbtp]
    \centering
    \includegraphics[width=.8\textwidth]{assets/relative_cooperation_rates}
    \caption{The relationship between cooperation rates in each tournament}
    \label{fig:relative_cooperation_rates}
\end{figure}

In the next section we investigate if a small modification to Axelrod's original
tournament would have changed the conclusions. Specifically: would a small
set of extra submissions changed the outcome?

\subsection{Running with extra invitations: Adding 1, 2 and 3 strategies}\label{sec:extra_strategy}

Using the full tournament results it is possible to extract results for adding
new strategies.

Table~\ref{tbl:original_tournament_with_extra_strategy_summary} shows the top
ranking strategies when adding a single strategy to the original tournament.

\begin{table}[!hbtp]
        \centering
        \begin{tabular}{lrrrr}
\toprule
{} &  Cooperation Rate against opponents &  Cooperation Rate from opponents &  Library Rank &  Rank \\
Name                       &                                     &                                  &               &       \\
\midrule
Omega TFT                  &                               0.888 &                            0.904 &            16 &     2 \\
Soft Joss                  &                               0.934 &                            0.926 &            48 &     2 \\
ZD-GTFT-2                  &                               0.941 &                            0.928 &            49 &     2 \\
Gradual                    &                               0.905 &                            0.911 &            20 &     2 \\
PSO Gambler 2\_2\_2 Noise 05 &                               0.866 &                            0.880 &            19 &     3 \\
Stein and Rapoport         &                               0.864 &                            0.882 &            44 &     7 \\
Spiteful Tit For Tat       &                               0.856 &                            0.876 &            29 &     8 \\
EugineNier                 &                               0.824 &                            0.857 &            31 &    12 \\
General Soft Grudger       &                               0.894 &                            0.879 &            51 &    14 \\
Shubik                     &                               0.831 &                            0.853 &            40 &    14 \\
EvolvedLookerUp2\_2\_2       &                               0.839 &                            0.842 &             1 &    14 \\
Evolved HMM 5              &                               0.831 &                            0.836 &             2 &    16 \\
Evolved FSM 16 Noise 05    &                               0.819 &                            0.832 &             5 &    22 \\
Winner12                   &                               0.754 &                            0.790 &            13 &    22 \\
PSO Gambler Mem1           &                               0.816 &                            0.824 &             9 &    24 \\
Evolved FSM 16             &                               0.812 &                            0.817 &             4 &    25 \\
Forgetful Fool Me Once     &                               0.812 &                            0.823 &            26 &    25 \\
Evolved ANN 5 Noise 05     &                               0.849 &                            0.838 &            11 &    26 \\
Evolved ANN 5              &                               0.828 &                            0.828 &             6 &    26 \\
PSO Gambler 2\_2\_2          &                               0.822 &                            0.836 &             3 &    26 \\
\bottomrule
\end{tabular}

        \caption{Performance of extra strategy in Axelrod's original tournament}
        \label{tbl:original_tournament_with_extra_strategy_summary}
\end{table}

Adding a single strategy from Table~\ref{tbl:original_tournament_with_extra_strategy_summary} 
would in fact not modify the outcome: Tit For Tat wins. In fact adding a
strategy either gives the same winner or gives either k42r or k60r as the
winner however, adding 2 or 3 strategies does change the outcome. 
Table~\ref{tbl:alternate_extra_strategy_tournament_winner_proportion} 
shows the proportion of times this happens.


\begin{table}[!hbtp]
        \centering
        \begin{tabular}{lrrr}
\toprule
 & 1 New (N =  209) & 2 New (N =  21,736) & 3 New (N =  1,499,784) \\
Winner &  &  &  \\
\midrule
Adaptive Tit For Tat: 0.5 & NaN & 0.000092 & 0.000232 \\
EugineNier: (D,) & NaN & NaN & 0.000001 \\
Firm But Fair & NaN & 0.000046 & 0.000123 \\
First by Stein and Rapoport: 0.05: (D, D) & NaN & NaN & 0.000001 \\
GTFT: 0.33 & NaN & NaN & 0.000020 \\
Gradual & NaN & NaN & 0.000001 \\
Omega TFT: 3, 8 & NaN & 0.000506 & 0.001457 \\
Original Gradual & NaN & 0.000046 & 0.000218 \\
PSO Gambler 2\_2\_2 Noise 05 & NaN & 0.000046 & 0.000227 \\
Resurrection & NaN & 0.000092 & 0.000256 \\
Soft Joss: 0.9 & NaN & 0.000092 & 0.000285 \\
Spiteful Tit For Tat & NaN & NaN & 0.000049 \\
ZD-GTFT-2: 0.25, 0.5 & NaN & 0.000138 & 0.000387 \\
k32r & NaN & NaN & 0.000007 \\
k41r & NaN & NaN & 0.000013 \\
k42r & 0.148325 & 0.269415 & 0.366403 \\
k44r & NaN & 0.000230 & 0.000568 \\
k49r & NaN & 0.000138 & 0.000347 \\
k60r & 0.004785 & 0.011180 & 0.018820 \\
k75r & NaN & 0.000506 & 0.002448 \\
k92r & 0.846890 & 0.717473 & 0.608137 \\
\bottomrule
\end{tabular}

        \caption{Proportion of winners of the original tournament with 1, 2 or 3
        new strategies}
        \label{tbl:alternate_extra_strategy_tournament_winner_proportion}
\end{table}

Clearly the more strategies added the less likely Tit For Tat is to win.
Although, interestingly the environment does not let the high performing
strategies of Table~\ref{tbl:full_tournament_rankings} to thrive. In the
next section we will investigate adding a particularly aggressive subset of
strategies.

\subsection{Running with extortion}\label{sec:run_with_stewart_plotkin}

Since the work of~\cite{Press2012} a lot of interest has been shown to Zero
Determinant strategies. In~\cite{Stewart2012} a small tournament is presented
pitting these against each other. Table~\ref{tbl:sp_tournament_rankings}
shows the rankings of the top 15 strategies when including all the Zero
Determinant strategies from~\cite{Stewart2012}.

\begin{table}[!hbtp]
        \centering
        \scriptsize
        \begin{tabular}{llrrrl}
\toprule
{} &                        Original Author &  Scores &  Rank &  Original Rank & Reproduced Rank \\
\midrule
ZD-GTFT-2 &                                     NA &  2.8366 &     1 &             NA &              NA \\
GTFT      &                                     NA &  2.8214 &     2 &             NA &              NA \\
k92r      &                        Anatol Rapoport &  2.8146 &     3 &              1 &               1 \\
k42r      &                          Otto Borufsen &  2.8046 &     4 &              3 &               2 \\
k75r      &                      Paul D Harrington &  2.7935 &     5 &              8 &               5 \\
k49r      &                               Rob Cave &  2.7894 &     6 &              4 &               4 \\
k44r      &                          William Adams &  2.7840 &     7 &              5 &               3 \\
k68r      &                       Fransois Leyvraz &  2.7717 &     8 &             12 &              10 \\
k41r      &                            Herb Weiner &  2.7648 &     9 &              7 &               8 \\
k32r      &                       Charles Kluepfel &  2.7617 &    10 &             10 &               6 \\
k46r      &                     Graham J Eatherley &  2.7614 &    11 &             14 &              13 \\
k72r      &                      Edward C White Jr &  2.7589 &    12 &             13 &              12 \\
k35r      &                        Abraham Getzler &  2.7506 &    13 &             11 &              11 \\
k84r      &  T Nicolaus Tideman and Paula Chieruzz &  2.7492 &    14 &              9 &               7 \\
k60r      &           Jim Graaskamp and Ken Katzen &  2.7382 &    15 &              6 &               9 \\
\bottomrule
\end{tabular}

        \caption{Top 15 strategies in the tournament composed of the original
                 strategies and the Zero Determinant strategies
                 from~\cite{Stewart2012}}
        \label{tbl:sp_tournament_rankings}
\end{table}

The overall cooperation rate of this tournament is
\input{assets/sp_tournament_overall_cooperation_rate.tex}and the various
cooperation rates are shown in
Figure~\ref{fig:sp_tournament_cooperation_rate_versus_rank} (ordered by rank).
The distribution is shown in Figure~\ref{fig:sp_tournament_cooperation_rates}.

\begin{figure}[!hbtp]
    \centering
    \includegraphics[width=.8\textwidth]{assets/sp_tournament_cooperation_rate_versus_rank.pdf}
    \caption{Cooperation rate versus rank for the Stewart and Poltkin tournament}
    \label{fig:sp_tournament_cooperation_rate_versus_rank}
\end{figure}


\begin{figure}[!hbtp]
    \centering
    \includegraphics[width=.8\textwidth]{assets/sp_tournament_cooperation_rates.pdf}
    \caption{Distribution of cooperation rates for the Stewart and Plotkin
    tournament}
    \label{fig:sp_tournament_cooperation_rates}
\end{figure}

Figure~\ref{fig:sp_relative_cooperation_rates} shows the relative
cooperation rates and here it identifies that the Zero Determinant strategies in
fact adapt their cooperation rate to the original tournament. They cooperate
more than.

\begin{figure}[!hbtp]
    \centering
    \includegraphics[width=.8\textwidth]{assets/sp_relative_cooperation_rates.pdf}
    \caption{The relationship between cooperation rates in each tournament}
    \label{fig:sp_relative_cooperation_rates}
\end{figure}

These results seem to highlight that adaptability is an important guideline for
cooperation...

\section{Conclusion}\label{sec:conclusion}

% TODO
% Describe what has been done:
% - Building axelrod_fortran, include references to best practice.
% - Highlight main conclusions: adding a single strategy would not change
%   outcomes but that in a more complex environment the conclusions no longer
%   apply
% - Discuss possibility of future work including reinforcement learning
%   techniques and more in depth evolutionary considerations.

\section*{Acknowledgements}

% TODO Write acknowledgements of software packages.

\bibliographystyle{plain}
\bibliography{bibliography.bib}

\appendix

\section{List of original players}\label{app:list_of_original_players}


\begin{multicols}{2}
    \begin{enumerate}
            \item k31r - Original rank: 23. Authored by Gail Grisell
\item k32r - Original rank: 10. Authored by Charles Kluepfel
\item k33r - Original rank: 59. Authored by Harold Rabbie
\item k34r - Original rank: 52. Authored by James W Friedman
\item k35r - Original rank: 11. Authored by Abraham Getzler
\item k36r - Original rank: 63. Authored by Roger Hotz
\item k37r - Original rank: 37. Authored by George Lefevre
\item k38r - Original rank: 34. Authored by Nelson Weiderman
\item k39r - Original rank: 25. Authored by Tom Almy
\item k40r - Original rank: 35. Authored by Robert Adams
\item k41r - Original rank: 7. Authored by Herb Weiner
\item k42r - Original rank: 3. Authored by Otto Borufsen
\item k43r - Original rank: 39. Authored by R D Anderson
\item k44r - Original rank: 5. Authored by William Adams
\item k45r - Original rank: 50. Authored by Michael F McGurrin
\item k46r - Original rank: 14. Authored by Graham J Eatherley
\item k47r - Original rank: 16. Authored by Richard Hufford
\item k48r - Original rank: 53. Authored by George Hufford
\item k49r - Original rank: 4. Authored by Rob Cave
\item k50r - Original rank: 54. Authored by Rik Smoody
\item k51r - Original rank: 18. Authored by John William Colbert
\item k52r - Original rank: 48. Authored by David A Smith
\item k53r - Original rank: 45. Authored by Henry Nussbacher
\item k54r - Original rank: 58. Authored by William H Robertson
\item k55r - Original rank: 42. Authored by Steve Newman
\item k56r - Original rank: 38. Authored by Stanley F Quayle
\item k57r - Original rank: 31. Authored by Rudy Nydegger
\item k58r - Original rank: 21. Authored by Glen Rowsam
\item k59r - Original rank: 40. Authored by Leslie Downing
\item k60r - Original rank: 6. Authored by Jim Graaskamp and Ken Katzen
\item k61r - Original rank: 2. Authored by Danny C Champion
\item k62r - Original rank: 51. Authored by Howard R Hollander
\item k63r - Original rank: 57. Authored by George Duisman
\item k64r - Original rank: 17. Authored by Brian Yamachi
\item k65r - Original rank: 47. Authored by Mark F Batell
\item k66r - Original rank: 20. Authored by Ray Mikkelson
\item k67r - Original rank: 27. Authored by Craig Feathers
\item k68r - Original rank: 12. Authored by Fransois Leyvraz
\item k69r - Original rank: 29. Authored by Johann Joss
\item k70r - Original rank: 32. Authored by Robert Pebly
\item k71r - Original rank: 60. Authored by James E Hall
\item k72r - Original rank: 13. Authored by Edward C White Jr
\item k73r - Original rank: 41. Authored by George Zimmerman
\item k74r - Original rank: 61. Authored by Edward Friedland
\item k75r - Original rank: 8. Authored by Paul D Harrington
\item k76r - Original rank: 46. Authored by David Gladstein
\item k77r - Original rank: 55. Authored by Scott Feld
\item k78r - Original rank: 19. Authored by Fred Mauk
\item k79r - Original rank: 26. Authored by Dennis Ambuehl and Kevin Hickey
\item k80r - Original rank: 36. Authored by Robyn M Dawes and Mark Batell
\item k81r - Original rank: 43. Authored by Martyn Jones
\item k82r - Original rank: 49. Authored by Robert A Leyland
\item k83r - Original rank: 15. Authored by Paul E Black
\item k84r - Original rank: 9. Authored by T Nicolaus Tideman and Paula Chieruzz
\item k85r - Original rank: 33. Authored by Robert B Falk and James M Langsted
\item k86r - Original rank: 28. Authored by Bernard Grofman
\item k87r - Original rank: 44. Authored by E E H Schurmann
\item k88r - Original rank: 22. Authored by Scott Appold
\item k89r - Original rank: 56. Authored by Gene Snodgrass
\item k90r - Original rank: 24. Authored by John Maynard Smith
\item k91r - Original rank: 30. Authored by Jonathan Pinkley
\item k92r - Original rank: 1. Authored by Anatol Rapoport
\item krandomc - Original rank: 62. Authored by None

    \end{enumerate}
\end{multicols}

\end{document}
